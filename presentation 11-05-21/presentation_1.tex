%%%%%%%%%%%%%%%%%%%%%%%%%%%%%%%%%%%%%%%%%
% Beamer Presentation
% LaTeX Template
% Version 1.0 (10/11/12)
%
% This template has been downloaded from:
% http://www.LaTeXTemplates.com
%
% License:
% CC BY-NC-SA 3.0 (http://creativecommons.org/licenses/by-nc-sa/3.0/)
%
%%%%%%%%%%%%%%%%%%%%%%%%%%%%%%%%%%%%%%%%%

%----------------------------------------------------------------------------------------
%	PACKAGES AND THEMES
%----------------------------------------------------------------------------------------

\documentclass{beamer}

\mode<presentation> {

% The Beamer class comes with a number of default slide themes
% which change the colors and layouts of slides. Below this is a list
% of all the themes, uncomment each in turn to see what they look like.

%\usetheme{default}
%\usetheme{AnnArbor}
%\usetheme{Antibes}
%\usetheme{Bergen}
%\usetheme{Berkeley}
%\usetheme{Berlin}
%\usetheme{Boadilla}
%\usetheme{CambridgeUS}
%\usetheme{Copenhagen}
%\usetheme{Darmstadt}
%\usetheme{Dresden}
%\usetheme{Frankfurt}
%\usetheme{Goettingen}
%\usetheme{Hannover}
%\usetheme{Ilmenau}
%\usetheme{JuanLesPins}
%\usetheme{Luebeck}
\usetheme{Madrid}
%\usetheme{Malmoe}
%\usetheme{Marburg}
%\usetheme{Montpellier}
%\usetheme{PaloAlto}
%\usetheme{Pittsburgh}
%\usetheme{Rochester}
%\usetheme{Singapore}
%\usetheme{Szeged}
%\usetheme{Warsaw}

% As well as themes, the Beamer class has a number of color themes
% for any slide theme. Uncomment each of these in turn to see how it
% changes the colors of your current slide theme.

%\usecolortheme{albatross}
%\usecolortheme{beaver}
%\usecolortheme{beetle}
%\usecolortheme{crane}
%\usecolortheme{dolphin}
%\usecolortheme{dove}
%\usecolortheme{fly}
%\usecolortheme{lily}
%\usecolortheme{orchid}
%\usecolortheme{rose}
%\usecolortheme{seagull}
%\usecolortheme{seahorse}
%\usecolortheme{whale}
%\usecolortheme{wolverine}

%\setbeamertemplate{footline} % To remove the footer line in all slides uncomment this line
%\setbeamertemplate{footline}[page number] % To replace the footer line in all slides with a simple slide count uncomment this line

%\setbeamertemplate{navigation symbols}{} % To remove the navigation symbols from the bottom of all slides uncomment this line
}

\usepackage{graphicx} % Allows including images
\usepackage{booktabs} % Allows the use of \toprule, \midrule and \bottomrule in tables

%----------------------------------------------------------------------------------------
%	TITLE PAGE
%----------------------------------------------------------------------------------------

\title[EMSE]{Implementation of Robot Behaviour Learning Simulator} % The short title appears at the bottom of every slide, the full title is only on the title page

\author{Kushagra Singh Bisen} % Your name
\institute[EMSE] % Your institution as it will appear on the bottom of every slide, may be shorthand to save space
{
Ecole des Mines de Saint Etienne \\ % Your institution for the title page
\medskip
\textit{kushagrasingh.bisen@etu.emse.fr} % Your email address
}
\date{\today} % Date, can be changed to a custom date

\begin{document}

\begin{frame}
\titlepage % Print the title page as the first slide
\end{frame}

\begin{frame}
    \frametitle{A Recap}
    In our last meeting,
    \begin{itemize}
        \item We discussed about the log file that was needed to be given as an output.
        \item Discussion regarding the Gazebo world was done.
        \item I was expected to present a simulation for the same.
    \end{itemize}
\end{frame}

\begin{frame}
    \frametitle{Today's Agenda}
    I have made a basic simulation, consisting of the following parts I will present after this slide.
    \begin{itemize}
        \item Development of Gazebo Grid Box, to our own personalization.
        \item An example of a teleop (controlling the robot with keyboard)
        \item A ROS Simulation Client.
        \item A ROS Simulation Server.
        \item A Potential log file, presenting the X and Y coorinates of the turtlebot.
    \end{itemize}
\end{frame}

\begin{frame}
    \frametitle{ROS Graph}
    \includegraphics[width=\textwidth]{rosgraph.png}
\end{frame}

\begin{frame}
    \frametitle{Work done on Friday/Monday}
    \begin{itemize}
        \item I started to work on a method that could make a seperate folder for the log files that are output in the terminal and then save it.
        \item I started learning the Simultaneous Localization and Mapping \textit{(SLAM)} algorithm for the robot to move around automatically and then provide a log file for the behaviour.
    \end{itemize}
\end{frame}

\begin{frame}
    \frametitle{Potential Work Goals}
    \begin{itemize}
        \item Simultaneous Localization and Mapping is a big part, and will take time.
        \item The current Gazebo world is not implemented in our own preference. Implementing SLAM in that Gazebo World.
        \item Learning and building Gazebo World Models with URDF file descriptions.
        \item The Log File's development. 
    \end{itemize}
\end{frame}

%----------------------------------------------------------------------------------------
%	PRESENTATION SLIDES
%----------------------------------------------------------------------------------------

\begin{frame}
\Huge{\centerline{Thank you!}}
\end{frame}

%----------------------------------------------------------------------------------------

\end{document} 