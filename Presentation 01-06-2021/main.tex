%%%%%%%%%%%%%%%%%%%%%%%%%%%%%%%%%%%%%%%%%
% Beamer Presentation
% LaTeX Template
% Version 1.0 (10/11/12)
%
% This template has been downloaded from:
% http://www.LaTeXTemplates.com
%
% License:
% CC BY-NC-SA 3.0 (http://creativecommons.org/licenses/by-nc-sa/3.0/)
%
%%%%%%%%%%%%%%%%%%%%%%%%%%%%%%%%%%%%%%%%%

%----------------------------------------------------------------------------------------
%	PACKAGES AND THEMES
%----------------------------------------------------------------------------------------

\documentclass{beamer}

\mode<presentation> {

% The Beamer class comes with a number of default slide themes
% which change the colors and layouts of slides. Below this is a list
% of all the themes, uncomment each in turn to see what they look like.

%\usetheme{default}
%\usetheme{AnnArbor}
%\usetheme{Antibes}
%\usetheme{Bergen}
%\usetheme{Berkeley}
%\usetheme{Berlin}
%\usetheme{Boadilla}
%\usetheme{CambridgeUS}
%\usetheme{Copenhagen}
%\usetheme{Darmstadt}
%\usetheme{Dresden}
%\usetheme{Frankfurt}
%\usetheme{Goettingen}
%\usetheme{Hannover}
%\usetheme{Ilmenau}
%\usetheme{JuanLesPins}
%\usetheme{Luebeck}
\usetheme{Madrid}
%\usetheme{Malmoe}
%\usetheme{Marburg}
%\usetheme{Montpellier}
%\usetheme{PaloAlto}
%\usetheme{Pittsburgh}
%\usetheme{Rochester}
%\usetheme{Singapore}
%\usetheme{Szeged}
%\usetheme{Warsaw}

% As well as themes, the Beamer class has a number of color themes
% for any slide theme. Uncomment each of these in turn to see how it
% changes the colors of your current slide theme.

%\usecolortheme{albatross}
%\usecolortheme{beaver}
%\usecolortheme{beetle}
%\usecolortheme{crane}
%\usecolortheme{dolphin}
%\usecolortheme{dove}
%\usecolortheme{fly}
%\usecolortheme{lily}
%\usecolortheme{orchid}
%\usecolortheme{rose}
%\usecolortheme{seagull}
%\usecolortheme{seahorse}
%\usecolortheme{whale}
%\usecolortheme{wolverine}

%\setbeamertemplate{footline} % To remove the footer line in all slides uncomment this line
%\setbeamertemplate{footline}[page number] % To replace the footer line in all slides with a simple slide count uncomment this line

%\setbeamertemplate{navigation symbols}{} % To remove the navigation symbols from the bottom of all slides uncomment this line
}

\usepackage{graphicx} % Allows including images
\usepackage{booktabs} % Allows the use of \toprule, \midrule and \bottomrule in tables

%----------------------------------------------------------------------------------------
%	TITLE PAGE
%----------------------------------------------------------------------------------------

\title[EMSE]{Implementation of a Robot Behavior Learning Simulator.} % The short title appears at the bottom of every slide, the full title is only on the title page

\author{Kushagra Singh Bisen} % Your name
\institute[EMSE] % Your institution as it will appear on the bottom of every slide, may be shorthand to save space
{
Ecole des Mines de Saint Etienne \\ % Your institution for the title page
\medskip
\textit{kushagrasingh.bisen@etu.emse.fr} % Your email address
}
\date{\today} % Date, can be changed to a custom date

\begin{document}

\begin{frame}
\titlepage % Print the title page as the first slide
\end{frame}

% \begin{frame}
% \frametitle{Overview} % Table of contents slide, comment this block out to remove it
% \tableofcontents % Throughout your presentation, if you choose to use \section{} and \subsection{} commands, these will automatically be printed on this slide as an overview of your presentation
% \end{frame}

\section{A recap from last meeting.}
\begin{frame}
    \frametitle{Recap}
    \begin{itemize}
        \item There was a discussion upon the log file and it's Implementation.
        \item I discussed about the path-planning algorithms and their usage in our work.
    \end{itemize}
\end{frame}

\section{Today's Agenda}
\begin{frame}
    \frametitle{Today's Agenda}
    Today we will see how the complete path planning done completely by the robot. I implemented a Dijkstra path planning algorithm, in which I used the CostMap2D to decide to cost.
\end{frame}

\section{Navigation in ROS}
\begin{frame}
    \frametitle{Navigation}
    Navigation in ROS is defined as process through which a robot is moved from the current node to the goal node. The step-by-step process for initiation of the planning is,
    \begin{itemize}
        \item Development of a World in Gazebo.
        \item SLAM (Localization of the Robot and the Mapping of the World).
        \item After SLAM is done, save the generated map. (Note : The map is generated as a PGM (portable graymap) file. The Map is very different to the World.)
        \item The Goal Node is given and the Robot uses the Dijkstra's Algorithm to find the shortest path towards the Goal.
    \end{itemize}
\end{frame}

\begin{frame}
    \frametitle{Navigation Stack}
    \includegraphics[width=\textwidth]{images/navigation.png}
\end{frame}

\begin{frame}
    \frametitle{Algorithm Implemented (Dijkstra)}
    As the algorithm is pretty famous in computer science for planning the path from a current node to the goal node.
    The algorithm is implemented in our case is,
    \begin{itemize}
        \item Initial Node has the g-cost of 0, and it is added to the open-list.
        \item Repeat the following while open-list is not empty: \begin{itemize}
            \item  Extract the node with the smallest g-cost from open-list and call it current-node
            \item  Mark it as visited by adding it into closed-list
        \end{itemize}
        \item If the current-node is the GOAL, trace back the path. Otherwise, we fimd the neighbours of the current node (3*3 grid neighbours as described by Prof. Nida's Comments)
        \item If the node has already been visited, skip it.
        \item If the node has not been visited and not in the open-list, check if the cost is less than the current cost we have. If yes, update the cost and the parent node.
        \item If the node has not been visited, set the cost, set the parent node and add it to the open-list.
    \end{itemize}
\end{frame}



\begin{frame}
    \frametitle{Initiating the Simulation.}
    \begin{block}{Starting the World in Gazebo}
        roslaunch unit2$\_$pp simulation$\_$unit2.launch
    \end{block}
    \begin{block}{Starting the Path Planning}
        roslaunch unit2$\_$pp unit2$\_$solution.launch
    \end{block}

    I will now proceed demonstrating the simulation.
\end{frame}

\begin{frame}
    \frametitle{Key Takeaways}
    \begin{itemize}
        \item With the previous 'obstacle-avoidance' simulation, I was not able to do the grid based modelling of the scenario as it is required.
        \item Due to this approach, generating log file is easier (I hope). Because, as in the algorithm the node calculates the path successively of the neighbours. I also think I will be able to make the use of /twist/ topic in ROS to output the Rotation.
        \item I tried to use the inherant csv file generation by ROS log file, but our usecase is different thus I will have to write a seperte node to subscribe to specific topics and then log it into a file.
        \item I have a better way to model the World with the boundaries and squares being exactly described the way they should be. I will continue working with the present map that I made for the current log file simulation.
        \item The present simulation is ready to be pushed to the Gitlab repository of EMSE.
    \end{itemize}
\end{frame}


\begin{frame}
\Huge{\centerline{Thank you for your time.}}
\end{frame}

%----------------------------------------------------------------------------------------

\end{document} 