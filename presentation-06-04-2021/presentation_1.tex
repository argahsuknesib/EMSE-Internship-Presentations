%%%%%%%%%%%%%%%%%%%%%%%%%%%%%%%%%%%%%%%%%
% Beamer Presentation
% LaTeX Template
% Version 1.0 (10/11/12)
%
% This template has been downloaded from:
% http://www.LaTeXTemplates.com
%
% License:
% CC BY-NC-SA 3.0 (http://creativecommons.org/licenses/by-nc-sa/3.0/)
%
%%%%%%%%%%%%%%%%%%%%%%%%%%%%%%%%%%%%%%%%%

%----------------------------------------------------------------------------------------
%	PACKAGES AND THEMES
%----------------------------------------------------------------------------------------

\documentclass{beamer}

\mode<presentation> {

% The Beamer class comes with a number of default slide themes
% which change the colors and layouts of slides. Below this is a list
% of all the themes, uncomment each in turn to see what they look like.

%\usetheme{default}
%\usetheme{AnnArbor}
% \usetheme{Antibes}
%\usetheme{Bergen}
%\usetheme{Berkeley}
% \usetheme{Berlin}
%\usetheme{Boadilla}
% \usetheme{CambridgeUS}
%\usetheme{Copenhagen}
%\usetheme{Darmstadt}
%\usetheme{Dresden}
% \usetheme{Frankfurt}
%\usetheme{Goettingen}
%\usetheme{Hannover}
%\usetheme{Ilmenau}
%\usetheme{JuanLesPins}
% \usetheme{Luebeck}
% \usetheme{Madrid}
%\usetheme{Malmoe}
% \usetheme{Marburg}
%\usetheme{Montpellier}
%\usetheme{PaloAlto}
% \usetheme{Pittsburgh}
% \usetheme{Rochester}
\usetheme{Singapore}
% \usetheme{Szeged}
% \usetheme{Warsaw}

% As well as themes, the Beamer class has a number of color themes
% for any slide theme. Uncomment each of these in turn to see how it
% changes the colors of your current slide theme.

%\usecolortheme{albatross}
%\usecolortheme{beaver}
%\usecolortheme{beetle}
%\usecolortheme{crane}
%\usecolortheme{dolphin}
%\usecolortheme{dove}
% \usecolortheme{fly}
%\usecolortheme{lily}
% \usecolortheme{orchid}
%\usecolortheme{rose}
%\usecolortheme{seagull}
% \usecolortheme{seahorse}
\usecolortheme{whale}
% \usecolortheme{wolverine}

\setbeamertemplate{footline} % To remove the footer line in all slides uncomment this line
%\setbeamertemplate{footline}[page number] % To replace the footer line in all slides with a simple slide count uncomment this line

%\setbeamertemplate{navigation symbols}{} % To remove the navigation symbols from the bottom of all slides uncomment this line
}

\usepackage{graphicx} % Allows including images
\usepackage{booktabs} % Allows the use of \toprule, \midrule and \bottomrule in tables

%----------------------------------------------------------------------------------------
%	TITLE PAGE
%----------------------------------------------------------------------------------------

\title[ROS Simulation]{Implementation of a robot behaviour learning simulator} % The short title appears at the bottom of every slide, the full title is only on the title page

\author{Kushagra Singh BISEN} % Your name
\institute[EMSE] % Your institution as it will appear on the bottom of every slide, may be shorthand to save space
{
Ecole des Mines Saint-Etienne \\ % Your institution for the title page
\medskip
\textit{kushagrasingh.bisen@etu.emse.fr} % Your email address
}
\date{\today} % Date, can be changed to a custom date

\begin{document}

\begin{frame}
\titlepage % Print the title page as the first slide
\end{frame}

\begin{frame}
\frametitle{Overview} % Table of contents slide, comment this block out to remove it
\tableofcontents % Throughout your presentation, if you choose to use \section{} and \subsection{} commands, these will automatically be printed on this slide as an overview of your presentation
\end{frame}

%----------------------------------------------------------------------------------------
%	PRESENTATION SLIDES
%----------------------------------------------------------------------------------------

%------------------------------------------------
\section{Robot Simulation} % Sections can be created in order to organize your presentation into discrete blocks, all sections and subsections are automatically printed in the table of contents as an overview of the talk
%------------------------------------------------

\subsection{A Recap} % A subsection can be created just before a set of slides with a common theme to further break down your presentation into chunks

\begin{frame}
\frametitle{A Recap}
The meeting was held at 01/04 and following tasks were discussed.
\begin{itemize}
    \item A discussion to decide the tools and language for the task.
    \item Framework which will provide the movement with other robots.
    \item Developing some sort of simulation with the turtlebots.
\end{itemize}
\end{frame}

%------------------------------------------------
\subsection{The tools empolyed}

\begin{frame}
\frametitle{Tools}
\begin{block}{Operating System}
The simulation is done using ROS(Robot Operating System) an open source robotics product, which isn't a full-fledged but provides a great
abstraction level over the hardware level. Thus, preventing the end user to reinventing the wheel. ROS is trusted by major companies and
labs as well as Professor Boissier in EMSE. ROS is majorly a network with servant and master being able to change positions and roles to publish 
or subscribe to information from other nodes. ROS has a seperate distribution for each version of Ubuntu. 
\end{block}
\end{frame}

\begin{frame}
\frametitle{Tools}
\begin{block}{Docker}
Containerization is important to isolate the environment variables to affect the project. As, the end result will run in a robot
which I think will be running with some sort of Arduino/Raspberry Pi chipset attached to it, contanerization will be great for deployment.
Currently, I was able to do small simulations with containers but I was not able to do the project in a docker container, as I am not experienced
in deployment much. I will learn by doing the projects.
\end{block} 
\end{frame}

\begin{frame}
\frametitle{Tools}
\begin{block}{Programming language}
    There is an \emph{option} to work with Python, C++, Java, Ruby in ROS. In our simulation, Python will work with the best being
    the one with the most support (official support from ROS) and being able to intetwine with C++ (also officially supported) if needed.
    Python is the chosen language for these simulations.
\end{block}
\begin{block}{Online ROS Platforms}
    There are online ROS platforms in which you can build and simulate ROS simulations.
\end{block}
\end{frame}

\section{Result}
\begin{frame}
    \frametitle{Simulation}
    I simulated an environment with 3 turtlebots. 
    \begin{itemize}
        \item I am using ROS's Navigation package.
        \item I wrote code to simulate the movement of these three turtlebots in relation to one another, but the bots were given different destinations.
        \item The code is written, but there is some bug in the launch file for navigation.
        \item Gazebo is great for 3D but, if you want to give a 2D goal, then RViz is the Simulator to go for. Note that, we can use both simulators hand in hand.
        \item I will now present my simulation results.
    \end{itemize}
\end{frame}

\begin{frame}
    \frametitle{Simulation}
    \includegraphics[width=\textwidth]{images/Gazebo-1.png}
\end{frame}

\begin{frame}
    \frametitle{Simulation}
    \includegraphics[width=\textwidth]{images/options-rviz-1.png}
\end{frame}

\begin{frame}
    \frametitle{Simulation}
    \includegraphics[width=\textwidth]{images/options-rviz-2.png}
\end{frame}

\begin{frame}
    \frametitle{Simulation}
    \includegraphics[width=\textwidth]{images/rviz-3.png}
\end{frame}

\begin{frame}
    \frametitle{Simulation}
    \includegraphics[width=\textwidth]{images/nav-rviz.png}
\end{frame}

\begin{frame}
\Huge{\centerline{Thank you!}}
\end{frame}

%----------------------------------------------------------------------------------------

\end{document} 