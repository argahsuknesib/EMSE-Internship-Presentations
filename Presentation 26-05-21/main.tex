%%%%%%%%%%%%%%%%%%%%%%%%%%%%%%%%%%%%%%%%%
% Beamer Presentation
% LaTeX Template
% Version 1.0 (10/11/12)
%
% This template has been downloaded from:
% http://www.LaTeXTemplates.com
%
% License:
% CC BY-NC-SA 3.0 (http://creativecommons.org/licenses/by-nc-sa/3.0/)
%
%%%%%%%%%%%%%%%%%%%%%%%%%%%%%%%%%%%%%%%%%

%----------------------------------------------------------------------------------------
%	PACKAGES AND THEMES
%----------------------------------------------------------------------------------------

\documentclass{beamer}

\mode<presentation> {

% The Beamer class comes with a number of default slide themes
% which change the colors and layouts of slides. Below this is a list
% of all the themes, uncomment each in turn to see what they look like.

%\usetheme{default}
%\usetheme{AnnArbor}
%\usetheme{Antibes}
%\usetheme{Bergen}
%\usetheme{Berkeley}
%\usetheme{Berlin}
%\usetheme{Boadilla}
% \usetheme{CambridgeUS}
%\usetheme{Copenhagen}
% \usetheme{Darmstadt}
%\usetheme{Dresden}
%\usetheme{Frankfurt}
% \usetheme{Goettingen}
%\usetheme{Hannover}
%\usetheme{Ilmenau}
%\usetheme{JuanLesPins}
%\usetheme{Luebeck}
\usetheme{Madrid}
%\usetheme{Malmoe}
%\usetheme{Marburg}
% \usetheme{Montpellier}
%\usetheme{PaloAlto}
%\usetheme{Pittsburgh}
% \usetheme{Rochester}
%\usetheme{Singapore}
% \usetheme{Szeged}
%\usetheme{Warsaw}

% As well as themes, the Beamer class has a number of color themes
% for any slide theme. Uncomment each of these in turn to see how it
% changes the colors of your current slide theme.

% \usecolortheme{albatross}
%\usecolortheme{beaver}
% \usecolortheme{beetle}
%\usecolortheme{crane}
%\usecolortheme{dolphin}
% \usecolortheme{dove}
%\usecolortheme{fly}
% \usecolortheme{lily}
%\usecolortheme{orchid}
%\usecolortheme{rose}
% \usecolortheme{seagull}
%\usecolortheme{seahorse}
% \usecolortheme{whale}
% \usecolortheme{wolverine}
% 
%\setbeamertemplate{footline} % To remove the footer line in all slides uncomment this line
%\setbeamertemplate{footline}[page number] % To replace the footer line in all slides with a simple slide count uncomment this line

%\setbeamertemplate{navigation symbols}{} % To remove the navigation symbols from the bottom of all slides uncomment this line
}

\usepackage{graphicx} % Allows including images
\usepackage{booktabs} % Allows the use of \toprule, \midrule and \bottomrule in tables

%----------------------------------------------------------------------------------------
%	TITLE PAGE
%----------------------------------------------------------------------------------------

\title[EMSE]{Implementation of a Robot Behaviour Simulator} % The short title appears at the bottom of every slide, the full title is only on the title page

\author{Kushagra Singh Bisen} % Your name
\institute[EMSE] % Your institution as it will appear on the bottom of every slide, may be shorthand to save space
{
Ecole des Mines de Saint Etienne \\ % Your institution for the title page
\medskip
\textit{kushagrasingh.bisen@etu.emse.fr} % Your email address
}
\date{\today} % Date, can be changed to a custom date

\begin{document}

\begin{frame}
\titlepage % Print the title page as the first slide
\end{frame}

\section{A Recap}
\subsection{Previous Method}
\subsubsection{Introduction}
\begin{frame}
    \frametitle{Previous Method}
    In the previous method I demonstrated the method for 'Obstacle Avoidance' in a Gazebo World. The turtlebot was able to set the angular velocities according to the obstacle. The turtlebot would then move to another cell/part of the Grid next.
\end{frame}
\subsubsection{Shortcomings}
\begin{frame}
    \frametitle{Shortcomings of the Previous Method}
\includegraphics[width=\textwidth]{images/heuristic.jpg}
\end{frame}
\begin{frame}
    \frametitle{Shortcomings of the Previous Method}
    \includegraphics[width=\textwidth]{images/shortcoming.jpg}
\end{frame}
\subsection{Solving the Shortcoming}
\subsubsection{How?}
\begin{frame}
    \frametitle{Solving the Shortcoming}
\begin{itemize}
    \item The problem should be solved with SLAM (Simultaneous Localization and Mapping)
    \item SLAM as I discussed in previous meeting is used for Localisation (deciding where the robot is) and Mapping (getting information about the Map World)
    \item But, pre-existing SLAM Method (go-to-goal) uses the most optimized algorithm. This will cause a problem in our application. I will explain it with a diagram in the next slide.
\end{itemize}
\end{frame}

\begin{frame}
    \frametitle{Path Planning.}
    \includegraphics[width=\textwidth]{images/SLAM-Path-Planning.jpg}
\end{frame}
\begin{frame}
    \frametitle{Path Planning - Problems}
    \begin{itemize}
        \item As it uses an optimized algorithm for path planning, the robot will employ various degrees of rotation.
        \item This practice is not useful for our application as we wish to employ just 4 MOVEMENTS not multiple.
        \item Thus, a planner node is to be written over the current SLAM procedure using either of these algorithms. \begin{itemize}
            \item A\textsuperscript{*}, or A algorithm.
            \item RRT (Rapidly Exploring Random Trees) or RRT\textsuperscript{*} algorithm.
            \item D or D* Algorithm.
        \end{itemize}
    \end{itemize}
\end{frame}

\begin{frame}
    \frametitle{Solution to Path Planning}
\begin{itemize}
    \item I have studied A\textsuperscript{*} algorithm in my Bachelor's Degree, and it is quite a common algorithm in Artificial Intelligence.
    \item I will try to employ the algorithm in a way to the present problem, it won't be A\textsuperscript{*} as maybe it won't be optimized.
    \item The way for path planning will be to write the movement and velocity with just 4 Movements.
    \item If the present algorithm is fine for the process. We will move with it, otherwise we will employ some other Algorithm.
\end{itemize}
\end{frame}

\begin{frame}
    \frametitle{Next Goal}
    \begin{itemize}
        \item Prepare a README.md file, elaborating the setup/basic things to setup a Turtlebot3 environment with a world in Gazebo in Professor Nida's Virtual Machine.
        \item Work on the planner algorithm for SLAM.
    \end{itemize}
\end{frame}

\begin{frame}
    \frametitle{Professor Mihaela's Comments.}
    I suggest in immediate time (this week, next week)
    \begin{itemize}
        \item - implement firstly a strategy (quite simple in 90° only but assuring that the goal will be reach in a finite time)
        \item - extract the logs. the logs have to be in a "cute format" (xml or json or other format semi-structured).
        \item - treat the logs to put in "Nida need format"
        \item - The logs will be analyzed with ML methods
        \item Than implement other strategies
    \end{itemize}

    

\end{frame}

\begin{frame}
\Huge{\centerline{Thank you for your time.}}
\end{frame}

\end{document} 