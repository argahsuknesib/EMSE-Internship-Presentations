%%%%%%%%%%%%%%%%%%%%%%%%%%%%%%%%%%%%%%%%%
% Beamer Presentation
% LaTeX Template
% Version 1.0 (10/11/12)
%
% This template has been downloaded from:
% http://www.LaTeXTemplates.com
%
% License:
% CC BY-NC-SA 3.0 (http://creativecommons.org/licenses/by-nc-sa/3.0/)
%
%%%%%%%%%%%%%%%%%%%%%%%%%%%%%%%%%%%%%%%%%

%----------------------------------------------------------------------------------------
%	PACKAGES AND THEMES
%----------------------------------------------------------------------------------------

\documentclass{beamer}

\mode<presentation> {

% The Beamer class comes with a number of default slide themes
% which change the colors and layouts of slides. Below this is a list
% of all the themes, uncomment each in turn to see what they look like.

%\usetheme{default}
%\usetheme{AnnArbor}
%\usetheme{Antibes}
%\usetheme{Bergen}
%\usetheme{Berkeley}
%\usetheme{Berlin}
%\usetheme{Boadilla}
%\usetheme{CambridgeUS}
%\usetheme{Copenhagen}
%\usetheme{Darmstadt}
%\usetheme{Dresden}
%\usetheme{Frankfurt}
%\usetheme{Goettingen}
%\usetheme{Hannover}
%\usetheme{Ilmenau}
%\usetheme{JuanLesPins}
%\usetheme{Luebeck}
\usetheme{Madrid}
%\usetheme{Malmoe}
%\usetheme{Marburg}
%\usetheme{Montpellier}
%\usetheme{PaloAlto}
%\usetheme{Pittsburgh}
%\usetheme{Rochester}
%\usetheme{Singapore}
%\usetheme{Szeged}
%\usetheme{Warsaw}

% As well as themes, the Beamer class has a number of color themes
% for any slide theme. Uncomment each of these in turn to see how it
% changes the colors of your current slide theme.

%\usecolortheme{albatross}
%\usecolortheme{beaver}
%\usecolortheme{beetle}
%\usecolortheme{crane}
%\usecolortheme{dolphin}
%\usecolortheme{dove}
%\usecolortheme{fly}
%\usecolortheme{lily}
%\usecolortheme{orchid}
%\usecolortheme{rose}
%\usecolortheme{seagull}
%\usecolortheme{seahorse}
%\usecolortheme{whale}
%\usecolortheme{wolverine}

%\setbeamertemplate{footline} % To remove the footer line in all slides uncomment this line
%\setbeamertemplate{footline}[page number] % To replace the footer line in all slides with a simple slide count uncomment this line

%\setbeamertemplate{navigation symbols}{} % To remove the navigation symbols from the bottom of all slides uncomment this line
}

\usepackage{graphicx} % Allows including images
\usepackage{booktabs} % Allows the use of \toprule, \midrule and \bottomrule in tables

%----------------------------------------------------------------------------------------
%	TITLE PAGE
%----------------------------------------------------------------------------------------

\title[ROS]{Implementation of Robot Behaviour Learning Simulator} % The short title appears at the bottom of every slide, the full title is only on the title page

\author{Kushagra Singh Bisen} % Your name
\institute[EMSE] % Your institution as it will appear on the bottom of every slide, may be shorthand to save space
{
Ecole des Mines de Saint Etienne \\ % Your institution for the title page
\medskip
\textit{kushagrasingh.bisen@etu.emse.fr} % Your email address
}
\date{\today} % Date, can be changed to a custom date

\begin{document}

\begin{frame}
\titlepage % Print the title page as the first slide
\end{frame}

% \begin{frame}
% \frametitle{Overview} % Table of contents slide, comment this block out to remove it
% \tableofcontents % Throughout your presentation, if you choose to use \section{} and \subsection{} commands, these will automatically be printed on this slide as an overview of your presentation
% \end{frame}

\begin{frame}
    \frametitle{A recap.}
    In the last meeting, we discussed a basic method for making the log file work. The problem we faced was related to the tumbling window we would require for the robot to move from the start position A and the end goal position B.
\end{frame}

\begin{frame}
    \frametitle{Agenda for Meeting.}
    Today, we will see a demo log file method that I tried to implement.
    \begin{block}{Method}
        I used the costmap2D library and split the whole map into small boxes with values between [0-255], these values define the 'free'ness of the space for the to move around. 
        I used python's list data structure and traversed the adjacent 8 neighbours of the robot in the current space and appended the list incase I had to add it's position and the type of space.
    \end{block}
\end{frame}

\begin{frame}
    \frametitle{Images.}
    \includegraphics[width=\textwidth]{images/crop-image-one.png}
    \includegraphics[width=\textwidth]{images/crop-image-two.png}
    \includegraphics[width=\textwidth]{images/crop-image-three.png}
\end{frame}

\begin{frame}
    \frametitle{Problems currently faced.}
    The logic used above was able to print the value repeatedly at each step but due to the irreugarities between the difference of end node position and start node position, there was an error in logic.
    I am also not certain if the increment by 1, with i = i + '1' (let me know if you wish to see the python logic) or not.
\end{frame}

\begin{frame}
    \frametitle{Further Work}
    \begin{itemize}
        \item I will work on how to iterate the value over the destination nodes.
        \item I have also made a plan[] list before in the process so maybe it will help me see log the movement (rotation) metric to the log file at each step (Not sure, but I will see)
        \item I will also try to check if the move base node can help us to iterate over the planned path.  
    \end{itemize}
\end{frame}


\begin{frame}
    \frametitle{RQT-GRAPH describing the nodes and topics involved.}
    \includegraphics[width=\textwidth]{images/rosgraph_active_nodes.png}
\end{frame}


\begin{frame}
\Huge{\centerline{Thank you for your time.}}
\end{frame}

%----------------------------------------------------------------------------------------

\end{document} 