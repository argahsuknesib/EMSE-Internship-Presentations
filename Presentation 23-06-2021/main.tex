%%%%%%%%%%%%%%%%%%%%%%%%%%%%%%%%%%%%%%%%%
% Beamer Presentation
% LaTeX Template
% Version 1.0 (10/11/12)
%
% This template has been downloaded from:
% http://www.LaTeXTemplates.com
%
% License:
% CC BY-NC-SA 3.0 (http://creativecommons.org/licenses/by-nc-sa/3.0/)
%
%%%%%%%%%%%%%%%%%%%%%%%%%%%%%%%%%%%%%%%%%

%----------------------------------------------------------------------------------------
%	PACKAGES AND THEMES
%----------------------------------------------------------------------------------------

\documentclass{beamer}

\mode<presentation> {

% The Beamer class comes with a number of default slide themes
% which change the colors and layouts of slides. Below this is a list
% of all the themes, uncomment each in turn to see what they look like.

%\usetheme{default}
%\usetheme{AnnArbor}
%\usetheme{Antibes}
%\usetheme{Bergen}
%\usetheme{Berkeley}
%\usetheme{Berlin}
%\usetheme{Boadilla}
%\usetheme{CambridgeUS}
%\usetheme{Copenhagen}
%\usetheme{Darmstadt}
%\usetheme{Dresden}
%\usetheme{Frankfurt}
%\usetheme{Goettingen}
%\usetheme{Hannover}
%\usetheme{Ilmenau}
%\usetheme{JuanLesPins}
%\usetheme{Luebeck}
\usetheme{Madrid}
%\usetheme{Malmoe}
%\usetheme{Marburg}
%\usetheme{Montpellier}
%\usetheme{PaloAlto}
%\usetheme{Pittsburgh}
%\usetheme{Rochester}
%\usetheme{Singapore}
%\usetheme{Szeged}
%\usetheme{Warsaw}

% As well as themes, the Beamer class has a number of color themes
% for any slide theme. Uncomment each of these in turn to see how it
% changes the colors of your current slide theme.

%\usecolortheme{albatross}
%\usecolortheme{beaver}
%\usecolortheme{beetle}
%\usecolortheme{crane}
%\usecolortheme{dolphin}
%\usecolortheme{dove}
%\usecolortheme{fly}
%\usecolortheme{lily}
%\usecolortheme{orchid}
%\usecolortheme{rose}
%\usecolortheme{seagull}
%\usecolortheme{seahorse}
%\usecolortheme{whale}
%\usecolortheme{wolverine}

%\setbeamertemplate{footline} % To remove the footer line in all slides uncomment this line
%\setbeamertemplate{footline}[page number] % To replace the footer line in all slides with a simple slide count uncomment this line

%\setbeamertemplate{navigation symbols}{} % To remove the navigation symbols from the bottom of all slides uncomment this line
}

\usepackage{graphicx} % Allows including images
\usepackage{booktabs} % Allows the use of \toprule, \midrule and \bottomrule in tables

%----------------------------------------------------------------------------------------
%	TITLE PAGE
%----------------------------------------------------------------------------------------

\title[Robot Simulation]{Implementation of a Robot Behaviour Learning Simulator} % The short title appears at the bottom of every slide, the full title is only on the title page

\author{Kushagra Singh Bisen} % Your name
\institute[EMSE] % Your institution as it will appear on the bottom of every slide, may be shorthand to save space
{
Ecole des Mines de Saint Etienne \\ % Your institution for the title page
\medskip
\textit{kushagrasingh.bisen@etu.emse.fr} % Your email address
}
\date{\today} % Date, can be changed to a custom date

\begin{document}

\begin{frame}
\titlepage % Print the title page as the first slide
\end{frame}

% \begin{frame}
% \frametitle{Overview} % Table of contents slide, comment this block out to remove it
% \tableofcontents % Throughout your presentation, if you choose to use \section{} and \subsection{} commands, these will automatically be printed on this slide as an overview of your presentation
% \end{frame}

%----------------------------------------------------------------------------------------
%	PRESENTATION SLIDES
%----------------------------------------------------------------------------------------

%------------------------------------------------

\begin{frame}
    \frametitle{Improvements Made}
    To be honest, I was not very productive this week. I apologize for that. I saw the collision avoidance methods and algorithms that I saw last week, but wasn't able to implement those algorithms. Moreover, I wasn't able to get the function up and running to print the log file for obstacle/free-space detection. I did however followed for some different approaches I can do multi robot path planning and have some questions for the same to ask after the end of the presentation.
\end{frame}

\begin{frame}
    \frametitle{Approaches to use.}
    As I wasn't able to implement those collision avoidance algorithms, I looked for various different approaches.
    \begin{itemize}
        \item Multi TurtleBot Simulation - This package is present with an argument for the number of turtlebots you want in the environment and it's starting position.
        \item Interceptor Simulation - One robot will move and the other will try to intercept it's path and then the particular collision avoidance will be done.
        \item Another Multiple Robot Simulator Package - which will convert the map into a search graph and then execute the path planning. (Attached paper in the mail)
    \end{itemize}
\end{frame}

\begin{frame}
    \frametitle{Simulation Video}
    I will show a simulation video for the multiple turtlebot simulation package now.
\end{frame}

\begin{frame}
    \frametitle{A request for help.}
    I'm not sure if the approach I am employing for generation of obstacle/free-space log file, and since I am not an expert in this topic it will be helpful if I could get some idea in ROS. I asked Prof. Boissier for help as he had a lecture about ROS in one of his lectures. He told me to contact someone named "Fabien Badeig" in April but I didn't get a reply. It will be helpful for me if it is possible by you to ask him to reply. (It is completely fine not to, I will search more on such methods with no worries).
    \includegraphics[width=0.9\textwidth]{images/image.png}
\end{frame}

\begin{frame}
    \frametitle{A question.}
    As we are trying to 'predict' the colliison in an environment in the end goal of the experiment. If this is the case, won't using a predefined algorithm with more than 99/100 accuracy of collision avoidance will be irrevelant for the simulation study we are doing?  
\end{frame}

\begin{frame}
\Huge{\centerline{Thank you for your time.}}
\end{frame}

%----------------------------------------------------------------------------------------

\end{document} 